\section*{Aufgabe 1: Wissenschaftstheorie}
\addcontentsline{toc}{section}{Aufgabe 1: Wissenschaftstheorie}

\begin{enumerate}
	\itshape{
	\item Suche Dir einen wissenschaftlichen Artikel aus dem Fachgebiet der IT und Technik aus (z.B. aus einer Fachzeitschrift, einem Fachbuch, Sammelband oder Conference Proceedings). Wähle dabei einen Artikel, der nicht im Skript erwähnt wird. Die Quelle muss die Kriterien für eine gute wissenschaftliche Arbeit erfüllen, um damit diese und die folgenden Workbook Aufgaben zu bearbeiten (z.B. typische Struktur, hinreichende Länge und Literaturverzeichnis mit mehreren Quellen). Nutze hierfür bevorzugt die Online-Bibliothek der Library and Information Services (LIS). Führe den Artikel mit vollständigen Literaturangaben gemäß den Regelungen aus dem Skript/Zitierleitfaden auf.
	\item Fasse
	      \begin{enumerate}
		      \item die Problemstellung/den Hintergrund des Artikels,
		      \item die Zielsetzung/Forschungsfrage(n)/Ziele und
		      \item die wesentlichen Ergebnisse und Schlussfolgerungen aus dem Artikel in eigenen Worten zusammen. Das heißt, Du sollst den Text indirekt zitieren, gemäß den Zitationsregeln aus dem Skript und Zitierleitfaden.
	      \end{enumerate}
	\item Bestimme die Forschungsmethodik aus dem Artikel und nenne drei Argumente, warum es sich um eine qualitative bzw. quantitative Forschungsmethodik oder bspw. um eine Literatur- oder Übersichtsarbeit handelt.}
\end{enumerate}

\clearpage

\subsection*{Bearbeitung Aufgabe 1: Wissenschaftstheorie}
\addcontentsline{toc}{subsection}{Bearbeitung Aufgabe 1: Wissenschaftstheorie}

\fullcite{steinmann2024}

Der Artikel behandelt die Herausforderung, wie \ac{KI} im Content Marketing optimal genutzt werden kann, um qualitativ hochwertige Texte zu erzeugen \parencite[S. 402-404]{steinmann2024}. Besonders wichtig ist dabei die menschliche Eingabe, der sogenannte Prompt, der maßgeblich die Qualität des erzeugten Textes beeinflusst. Nach Steinmann und Piazza fehlt es aktuell an klaren Anleitungen und Techniken, um durch gezieltes Prompting die Qualität der KI-generierten Texte zu gewährleisten, sodass sie den Anforderungen im Content Marketing entsprechen.

\textcite[S.404]{steinmann2024} untersuchen, wie sich die Textausgabe von ChatGPT durch gezieltes Prompt Engineering steuern lässt, um die Qualitätskriterien im Content Marketing zu erfüllen. Im Kern ihrer Forschung steht die Frage, welche Prompt-Strukturen und -Techniken besonders effektiv sind, um hochwertige Marketingtexte zu generieren (S.402, 407).

Wesentlichen Ergebnisse der Studie zeigen, dass die Techniken des Zero-shot Chain-of-Thought und des One-shot Promptings besonders effektiv genutzt werden können, um die Qualität der von ChatGPT generierten Texte zu verbessern \parencite[S. 412]{steinmann2024}. Diese Prompting-Techniken ermöglichen eine gezielte Steuerung des \ac{KI}-generierten Outputs in Richtung der definierten Erfolgskriterien für Content Marketing Texte. Zudem werden die gegenwärtigen Schwächen von \ac{KI}-generierten Texten wie mangelnde Emotionalität und stilistische Mängel aufgezeigt. Die Schlussfolgerung ist, dass eine kollaborative Wertschöpfung von Mensch und \ac{KI} erforderlich ist, um die Qualität der generierten Texte zu maximieren und die definierten Qualitätsziele zu erreichen (S. 413).

Die Methodik der Forschung von \textcite[S. 406-412]{steinmann2024} basiert auf der \ac{DSRM}. In dieser Arbeit  wird die \ac{DSRM} überwiegend qualitativ verwendet. Drei Argumente, warum es sich hierbei um eine qualitative Forschungsmethodik handelt, sind:
\begin{enumerate}
	\item Die Erstellung eines nicht quantifizierbaren Artefakts (effektive Prompt-Struktur), dessen Qualität durch die Einschätzungen der generierten Texte beurteilt wird.
	\item Die Evaluation der durch die KI generierten Texte erfolgt durch Interviews mit Experten, welche ihre Einschätzung zum generierten Text geben.
	\item Der iterative Prozess der Entwicklung und Anpassung des Prompts basiert auf kontinuierlichem qualitativem Feedback, anstatt auf quantitativen Messungen.
\end{enumerate}

Diese Argumente unterstreichen die qualitative Natur der hier gebrauchten Methodik.
