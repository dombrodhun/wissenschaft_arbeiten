\addsec{Aufgabe 1: Wissenschaftstheorie}

\begin{enumerate}[leftmargin=*]
	\itshape
	\item Suche Dir einen wissenschaftlichen Artikel aus dem Fachgebiet der IT und Technik aus (z.B. aus einer Fachzeitschrift, einem Fachbuch, Sammelband oder Conference Proceedings). Wähle dabei einen Artikel, der nicht im Skript erwähnt wird. Die Quelle muss die Kriterien für eine gute wissenschaftliche Arbeit erfüllen, um damit diese und die folgenden Workbook Aufgaben zu bearbeiten (z.B. typische Struktur, hinreichende Länge und Literaturverzeichnis mit mehreren Quellen). Nutze hierfür bevorzugt die Online-Bibliothek der \gls{lis}. Führe den Artikel mit vollständigen Literaturangaben gemäß den Regelungen aus dem Skript/Zitierleitfaden auf.
	\item Fasse
	      \begin{enumerate}[label=\arabic*.]
		      \item die Problemstellung/den Hintergrund des Artikels,
		      \item die Zielsetzung/Forschungsfrage(n)/Ziele und
		      \item die wesentlichen Ergebnisse und Schlussfolgerungen aus dem Artikel in eigenen Worten zusammen. Das heißt, Du sollst den Text indirekt zitieren, gemäß den Zitationsregeln aus dem Skript und Zitierleitfaden.
	      \end{enumerate}
	\item Bestimme die Forschungsmethodik aus dem Artikel und nenne drei Argumente, warum es sich um eine qualitative bzw. quantitative Forschungsmethodik oder bspw. um eine Literatur- oder Übersichtsarbeit handelt.
\end{enumerate}
\upshape

\clearpage

\subsection*{Bearbeitung Aufgabe 1: Wissenschaftstheorie}
\addcontentsline{toc}{subsection}{Bearbeitung Aufgabe 1: Wissenschaftstheorie}

\fullcitebib{tambon2025}

\textcite[1-2]{tambon2025} argumentieren, dass von \glspl{llm} generierter Code erfolgversprechend genutzt wird, allerdings auch Fehler enthält. Die Eigenschaften und Muster der \gls{llm}-Fehler bei der Codegenerierung sind bislang kaum untersucht. Es ist unklar, wie sich die \gls{llm}-Fehler von menschlichen Fehlern im Code unterscheiden.

\textcite[3]{tambon2025} gehen auf zwei Forschungsfragen präziser ein:
\begin{enumerate}
	\item \enquote{What are the characteristics of bugs occurring in code generated by LLMs for real-world project tasks?}
	\item \enquote{To what extent are the identified bug patterns in LLM-generated code relevant for software practitioners and researchers working with LLMs?}
\end{enumerate}
Durch die Beantwortung wird sich ein besseres Verständnis von durch \gls{llm} generierten Fehlern versprochen, um die Qualitätssicherung in dem Bereich zu unterstützen.

Das Ergebnis der Codeanlyse sind zehn distinkte Fehlermuster, die in einer Taxonomie organisiert werden \parencite[12]{tambon2025}. Diese Fehlermuster werden in einer Umfrage bestätigt. Aus der Umfrage gehen ebenfalls drei weitere Fehlergruppen hervor, die für zukünftige Forschung wichtig werden könnten. Unter den Fehlermustern finden sich für den Menschen eher untypische, z.B. halluzinierte Objekte. Moderne \glspl{ide} unterstützen bei der Auffindnug und der Korrektur einiger Fehlermustern, aber Fehler wie \gls{npc} sind schwer zu erkennen, da sie Augenscheinlich korrekt sind.

Bei der empirischen Studie von \textcite{tambon2025} handelt es sich um eine Methodentriangulation:
\begin{enumerate}
	\item Die Autoren nutzten die Open Coding Methodologie, bei der sie Fehler im Code analysierten und kategorisierten. Dieser Prozess ist rein qualitativ.
	\item Die Autoren analysierten die Häufigkeitsverteilungen der Fehler über verschiedene \glspl{llm} und nutzten statistische Tests (Fisher-Exakt-Test). Hierbei handelt es sich um quantitative Forschung.
	\item Sie führten zwei Umfragen durch, um Ihre Ergebnisse zu validieren und statistische Analysen durchzuführen. Hierzu nutzten sie sowohl qualitative, als auch quantitative Methoden.
\end{enumerate}
Die Methodentriangulation unterstützt die Forschenden die Vorteile beider Methodiken für Ihre Arbeit zu nutzen.

