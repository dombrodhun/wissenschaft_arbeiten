\section*{Aufgabe 1: Wissenschaftstheorie}

\begin{enumerate}
	\itshape{
	\item Suche Dir einen wissenschaftlichen Artikel aus dem Fachgebiet der IT und Technik aus (z.B. aus einer Fachzeitschrift, einem Fachbuch, Sammelband oder Conference Proceedings). Wähle dabei einen Artikel, der nicht im Skript erwähnt wird. Die Quelle muss die Kriterien für eine gute wissenschaftliche Arbeit erfüllen, um damit diese und die folgenden Workbook Aufgaben zu bearbeiten (z.B. typische Struktur, hinreichende Länge und Literaturverzeichnis mit mehreren Quellen). Nutze hierfür bevorzugt die Online-Bibliothek der Library and Information Services (LIS). Führe den Artikel mit vollständigen Literaturangaben gemäß den Regelungen aus dem Skript/Zitierleitfaden auf.
	\item Fasse
	      \begin{enumerate}
		      \item die Problemstellung/den Hintergrund des Artikels,
		      \item die Zielsetzung/Forschungsfrage(n)/Ziele und
		      \item die wesentlichen Ergebnisse und Schlussfolgerungen aus dem Artikel in eigenen Worten zusammen. Das heißt, Du sollst den Text indirekt zitieren, gemäß den Zitationsregeln aus dem Skript und Zitierleitfaden.
	      \end{enumerate}
	\item Bestimme die Forschungsmethodik aus dem Artikel und nenne drei Argumente, warum es sich um eine qualitative bzw. quantitative Forschungsmethodik oder bspw. um eine Literatur- oder Übersichtsarbeit handelt.}
\end{enumerate}
