\addsec[Aufgabe 4: Anwendung guter wissenschaftlicher Praxis II]{Aufgabe 4: Anwendung guter wissenschaftlicher Praxis II: Einleitung und Forschungsfragen}

\begin{enumerate}[leftmargin=*]
	\itshape{
	\item Schreibe eine Einleitung für Deine wissenschaftliche Arbeit aus Aufgabe 3. Denke dabei an die
	      Bestandteile einer Einleitung und formuliere mindestens zwei Forschungsfragen.
	\item Zitiere mindestens eine Quelle direkt und zwei weitere indirekt in der Einleitung. Achte dabei darauf, dass
	      Du die Zitationsregeln aus dem Zitierleitfaden verwendest. Gerne kannst Du die Quellen nutzen, die Du
	      mittels Deiner Recherche aus Aufgabe 3 bereits für Dein Thema gefunden hast.
	      }
\end{enumerate}

\clearpage

\subsection*{Bearbeitung Aufgabe 4: Anwendung guter wissenschaftlicher Praxis II: Einleitung und Forschungsfragen}
\addcontentsline{toc}{subsection}{Aufgabe 4: Anwendung guter wissenschaftlicher Praxis II}
Der globale Zahlungsverkehr verlagert sich, angetrieben durch Open Banking, stetig zu digitalen Transaktionen \parencites{bis2025}[S. 12]{capgemini2024}. Die strategische Bedeutung dieses Wandels ist hoch, denn: \enquote{Beyond e-commerce, more than 65\% of payment executives recognize the significance of expanding instant payment infrastructure to drive noncash transactions} \parencite[S. 10]{capgemini2024}. Diese Entwicklung macht die intelligente Verarbeitung von Transaktionsdaten zu einer zentralen Herausforderung.

Manuelle und regelbasierte Ansätze zur Kategorisierung von Banktransaktionen sind ineffizient und fehleranfällig \parencite[S. 35-37]{capgemini2024}. Während \gls{ml} seine Überlegenheit in verwandten Aufgaben bereits bewiesen hat \parencite[S. 10]{cronin2017}, hat die spezifische Anwendung auf Banktransaktionen bisher nur begrenzte Aufmerksamkeit erhalten \parencite[S. 1-2]{ta2023}. Es fehlt somit eine systematische Vergleichsanalyse, um die am besten geeigneten \gls{ml}-Algorithmen für diese spezielle Aufgabe zu identifizieren.

Das Ziel dieser Arbeit ist die systematische Untersuchung und vergleichende Analyse ausgewählter \gls{ml}-Algorithmen im Hinblick auf ihre Eignung zur automatischen Kategorisierung von Banktransaktionen. Anhand definierter Gütekriterien soll die Leistungsfähigkeit der Modelle evaluiert werden, um eine fundierte Empfehlung für die Auswahl eines geeigneten Algorithmus für diese Problemstellung auszusprechen.

Um dieses Ziel zu erreichen, wird die folgende zentrale Forschungsfrage beantwortet:

\textbf{Welche \gls{ml}-Algorithmen eignen sich für die automatische Kategorisierung von Banktransaktionen und wie unterscheiden sie sich hinsichtlich ihrer Klassifikationsgüte?}

Zur Beantwortung werden die folgenden Unterfragen untersucht:

\begin{itemize}
	\item Wie leistungsfähig sind die ausgewählten \gls{ml}-Algorithmen bei der Klassifikation der aufbereiteten Daten, gemessen an den definierten Gütekriterien?
	\item Welcher der verglichenen Algorithmen bietet auf Basis der Ergebnisse die beste Klassifikationsgüte für den Anwendungsfall?
\end{itemize}

Die vorliegende Arbeit gliedert sich in sechs Kapitel. Zunächst werden in Kapitel 2 die theoretischen Grundlagen zu Textklassifikation und den ausgewählten \gls{ml}-Algorithmen erläutert. Kapitel 3 beschreibt die Methodik, einschließlich der Datenvorverarbeitung und des experimentellen Aufbaus. Darauf aufbauend werden in Kapitel 4 die Ergebnisse präsentiert und in Kapitel 5 diskutiert. Abschließend fasst Kapitel 6 die zentralen Erkenntnisse zusammen und gibt einen Ausblick auf weiterführende Forschung.
