\addsec{Aufgabe 5: Forschungsmethoden}

\begin{enumerate}[leftmargin=*]
	\itshape{
	\item Bestimme die geeignete Forschungsmethodik für Deine wissenschaftliche Arbeit aus Aufgabe 3 und
	      beschreibe die passende Methodik. Was willst Du entwickeln, welche Experimente willst Du durchführen,
	      welche Daten erheben? Du sollst nur die Erhebungs- und Auswertungsmethode \uline{beschreiben}. Es ist nicht
	      nötig, dass Du einen Prüfplan, Fragebogen oder ein anderes Instrument entwirfst.
	\item Begründe, warum die gewählte Forschungsmethodik geeignet für die Beantwortung Deiner
	      Forschungsfrage(n) ist.
	      }
\end{enumerate}

\clearpage

\subsection*{Bearbeitung Aufgabe 5: Forschungsmethoden}
\addcontentsline{toc}{subsection}{Bearbeitung Aufgabe 5: Forschungsmethoden}

Für die Vergleichsanalyse von \acrlong{ml}-Algorithmen zur Banktransaktions-Kategorisierung wird ein quantitativ-experimenteller Ansatz gewählt. Dieser Ansatz ist darauf ausgelegt, die Leistung verschiedener Modelle objektiv zu messen und zu vergleichen.

Die Untersuchung folgt einer standardisierten Pipeline, um reproduzierbare Ergebnisse zu gewährleisten.

\textbf{Entwicklung und Experiment:}
Es wird eine Verarbeitungspipeline in Python entwickelt. Diese beinhaltet die Datenvorverarbeitung (Tokenisierung, Stoppwort-Entfernung, Kleinschreibung) und die Merkmalsextraktion mittels \gls{tfidf}, um Textbeschreibungen in numerische Vektoren zu überführen. Anschließend werden etablierte Klassifikationsalgorithmen wie \gls{svm}, \gls{nb} und Ensemble-Methoden (\gls{rf}, \gls{catboost}) auf einem identischen Datensatz trainiert und mittels Kreuzvalidierung getestet.

\textbf{Datenerhebung und Auswertung:}
Aufgrund von Datenschutzauflagen wird ein öffentlicher, anonymisierter Datensatz (z.B. von Kaggle) oder ein synthetischer Datensatz verwendet. Die Modellevaluierung erfolgt quantitativ anhand der Metriken Precision, Recall und F1-Score. Wegen der ungleichen Kategorienverteilung wird der Makro-Mittelwert genutzt. Diese Metrik stellt sicher, dass alle Kategorien gleich gewichtet werden und die Leistung bei selteneren Klassen nicht von den häufigen Klassen überdeckt wird. Eine Konfusionsmatrix dient der Detailanalyse.

Der gewählte Ansatz ist ideal für dieses Forschungsvorhaben:

\textbf{Zielgerichtet:} Ein experimenteller Vergleich liefert eine direkte und empirisch gestützte Antwort auf die zentrale Forschungsfrage, welcher Algorithmus die beste Leistung für die gegebene Aufgabe erbringt.

\textbf{Objektiv und Reproduzierbar:} Die Verwendung standardisierter, numerischer Metriken ermöglicht einen objektiven Leistungsvergleich der Modelle. Die Methodik ist transparent und kann von Dritten nachvollzogen werden.

\textbf{Wissenschaftlich fundiert:} Das Vorgehen orientiert sich am etablierten Stand der Technik, wie er in den als Referenz dienenden wissenschaftlichen Arbeiten von \textcite{allegue2023, garcia-mendez2020, kotios2022, ta2023} demonstriert wird.

\textbf{Realistisch:} Die Implementierung und Evaluation existierender Algorithmen ist im vorgegebenen Zeitrahmen von elf Wochen machbar.
