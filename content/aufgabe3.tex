\addsec{Aufgabe 3: Anwendung guter wissenschaftlicher Praxis I: Themenfindung und Datenbankrecherche}

\begin{enumerate}
	\itshape{
	\item Überlege Dir ein Thema für eine wissenschaftliche Arbeit aus Deinem Fach- bzw. Studiengebiet, also zum Beispiel für eine Seminar-, Projekt- oder Abschlussarbeit. Formuliere einen eigenen aussagekräftigen Titel für Deine wissenschaftliche Arbeit.
	\item Nun sollst Du zu Deinem Thema passende Literatur finden.
	      \begin{enumerate}
		      \item Benenne fünf Suchbegriffe in Deutsch oder Englisch, die zu Deinem Thema passen und führe eine Recherche bevorzugt in der Online-Bibliothek der Library and Information Services (LIS) aus. Nutze hierbei mindestens einmal auch die booleschen Operatoren. Führe die Suchbegriffe in einer Tabelle auf.
		      \item Führe fünf wissenschaftliche Quellen in einem Literaturverzeichnis auf, die Du mittels der Recherche gefunden hast. Erstelle das Literaturverzeichnis gemäß den Regelungen, die Du im Zitierleitfaden findest.
	      \end{enumerate}}
\end{enumerate}

\clearpage

\subsection*{Bearbeitung Aufgabe 3: Anwendung guter wissenschaftlicher Praxis I: Themenfindung und Datenbankrecherche}
\addcontentsline{toc}{subsection}{Bearbeitung Aufgabe 3: Anwendung guter wissenschaftlicher Praxis I: Themenfindung und Datenbankrecherche}

