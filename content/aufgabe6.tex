\addsec{Aufgabe 6: Verzeichnisse erstellen und Planung}

\begin{enumerate}[leftmargin=*]
	\itshape{
	\item Erstelle eine Gliederung für Deine wissenschaftliche Arbeit aus Aufgabe 3. Beachte hierbei den Aufbau
	      einer wissenschaftlichen Arbeit und die Kapitel bzw. Verzeichnisse, die zwingend zum Aufbau einer
	      wissenschaftlichen Arbeit gehören.
	\item Erstelle ein Literaturverzeichnis für \uline{alle} in Deinem Workbook zitierten Quellen, also alle Quellen, die Du
	      in Aufgabe 1–5 genutzt hast, gemäß den Regelungen im Skript/Zitierleitfaden.
	\item Erstelle einen groben Zeitplan für das gewählte Projekt.
	      }
\end{enumerate}

\clearpage

\subsection*{Bearbeitung Aufgabe 6: Verzeichnisse erstellen und Planung}
\addcontentsline{toc}{subsection}{Bearbeitung Aufgabe 6: Verzeichnisse erstellen und Planung}

\noindent\textbf{Abbildungsverzeichnis}
\par\noindent\textbf{Tabellenverzeichnis}
\par\noindent\textbf{Abkürzungsverzeichnis}
\begin{enumerate}[leftmargin=*, label=\arabic*, font=\bfseries, noitemsep]
	\item\textbf{Einleitung}
	      \begin{enumerate}[label=\theenumi.\arabic*, noitemsep]
		      \item Hinführung zum Thema und Relevanz
		      \item Problemstellung
		      \item Zielsetzung und Forschungsfragen
		      \item Aufbau der Arbeit
	      \end{enumerate}
	\item\textbf{Grundlagen}
	      \begin{enumerate}[label=\theenumi.\arabic*, noitemsep]
		      \item Grundlagen der Textklassifikation
		      \item Vorstellung der ausgewählten Machine-Learning-Algorithmen
		            \begin{enumerate}[label=\theenumii.\arabic*, noitemsep]
			            \item Support Vector Machine (SVM)
			            \item Naive Bayes (NB)
			            \item Ensemble-Methoden (Random Forest, CatBoost)
		            \end{enumerate}
		      \item Evaluationsmetriken
	      \end{enumerate}
	\item\textbf{Methodik}
	      \begin{enumerate}[label=\theenumi.\arabic*, noitemsep]
		      \item Forschungsdesign und Vorgehen
		      \item Datengrundlage und Datenvorverarbeitung
		      \item Experimenteller Aufbau und Implementierung
	      \end{enumerate}
	\item\textbf{Ergebnisse}
	      \begin{enumerate}[label=\theenumi.\arabic*, noitemsep]
		      \item Darstellung der Klassifikationsergebnisse
		      \item Vergleichende Analyse der Algorithmen
	      \end{enumerate}
	\item\textbf{Diskussion}
	      \begin{enumerate}[label=\theenumi.\arabic*, noitemsep]
		      \item Interpretation der Ergebnisse
		      \item Beantwortung der Forschungsfragen
		      \item Limitationen der Arbeit
	      \end{enumerate}
	\item\textbf{Fazit und Ausblick}
	      \begin{enumerate}[label=\theenumi.\arabic*, noitemsep]
		      \item Zusammenfassung der Erkenntnisse
		      \item Ausblick auf weiterführende Forschung
	      \end{enumerate}
\end{enumerate}
\noindent\textbf{Literaturverzeichnis}
\par\noindent\textbf{Anhang}
