\textbf{Phase 1: Vorbereitung und Setup (Woche 1)}
\begin{itemize}[leftmargin=*, noitemsep]
	\item\textbf{Ziel:} Basis schaffen und detailliert planen
	\item\textbf{Aufgaben:}
	      \begin{itemize}[noitemsep]
		      \item Themenabstimmung: Finale Besprechung des Themas und der Forschungsfragen  mit deinem Betreuer.
		      \item Literaturvertiefung: Systematische Literaturrecherche vertiefen und Quellen in Zotero erfassen.
		      \item Entwicklungsumgebung: Projekt in Python aufsetzen, notwendige Bibliotheken (scikit-learn, pandas, CatBoost) installieren.
	      \end{itemize}
\end{itemize}

\textbf{Phase 2: Daten und Methodik (Woche 2-3)}
\begin{itemize}[leftmargin=*, noitemsep]
	\item\textbf{Ziel:} Datengrundlage schaffen und das experimentelle Vorgehen finalisieren.
	\item\textbf{Aufgaben:}
	      \begin{itemize}[noitemsep]
		      \item Woche 2: Datensatz \& Vorverarbeitung: Geeigneten öffentlichen Datensatz (z.B. von Kaggle) finden und analysieren. Start der Implementierung der Datenvorverarbeitung (Tokenisierung, Stoppwort-Entfernung etc.).
		      \item Woche 3: Schreibarbeit \& Implementierung: Kapitel 2 (Grundlagen) und Kapitel 3 (Methodik) im Rohentwurf schreiben.  Parallel dazu die Verarbeitungspipeline in Python fertigstellen.
	      \end{itemize}
\end{itemize}

\textbf{Phase 3: Experimente und erste Ergebnisse (Woche 4-6)}
\begin{itemize}[leftmargin=*, noitemsep]
	\item\textbf{Ziel:} Modelle implementieren und erste Ergebnisse generieren.
	\item\textbf{Aufgaben:}
	      \begin{itemize}[noitemsep]
		      \item Woche 4: Implementierung der Modelle: Die Klassifikationsalgorithmen (SVM, NB, RF, CatBoost) implementieren.
		      \item Woche 5: Experimente durchführen: Die Modelle trainieren, Kreuzvalidierung anwenden und Ergebnisse (Precision, Recall, F1-Score) systematisch speichern.
		      \item Woche 6: Ergebnisse aufbereiten \& schreiben: Die Ergebnisse in Tabellen und Grafiken aufbereiten. Kapitel 4 (Ergebnisse) verfassen.
	      \end{itemize}
\end{itemize}

\textbf{Phase 4: Kern-Schreibphase (Woche 7-9)}
\begin{itemize}[leftmargin=*, noitemsep]
	\item\textbf{Ziel:} Den Großteil der Arbeit verschriftlichen und die Ergebnisse interpretieren.
	\item\textbf{Aufgaben:}
	      \begin{itemize}[noitemsep]
		      \item Woche 7: Diskussion: Kapitel 5 (Diskussion) schreiben. Hier interpretierst du deine Ergebnisse und beantwortest die Forschungsfragen.
		      \item Woche 8: Einleitung \& Fazit: Kapitel 1 (Einleitung) und Kapitel 6 (Fazit und Ausblick) schreiben.
		      \item Woche 9: Überarbeitung: Den gesamten Text auf inhaltliche Stimmigkeit und einen roten Faden prüfen. Alle Verzeichnisse (Literatur, Abbildungen etc.) erstellen und finalisieren.
	      \end{itemize}
\end{itemize}

\textbf{Phase 5: Finale Überarbeitung und Abgabe (Woche 10-11)}
\begin{itemize}[leftmargin=*, noitemsep]
	\item\textbf{Ziel:} Die Arbeit finalisieren und abgeben.
	\item\textbf{Aufgaben:}
	      \begin{itemize}[noitemsep]
		      \item Woche 10: Korrekturphase: Komplette Arbeit auf Rechtschreibung, Grammatik und Zitierregeln prüfen. Idealerweise von einer anderen Person gegenlesen lassen.
		      \item Woche 11: Puffer \& Abgabe: Letzte Korrekturen einarbeiten, Formatierung prüfen, Druck und Bindung organisieren. Diese Woche dient als Puffer für unvorhergesehene Verzögerungen.
	      \end{itemize}
\end{itemize}


